

% Clase del documento.
\documentclass[a4paper,12pt,spanish]{article} %agregar twoside para impresion Doble Faz.
\usepackage{a4wide} %Importante que este arriba asi todo lo demas se acomoda a estos margenes.

% Paquetes
\usepackage[spanish, activeacute]{babel} %Paquetes para que funcionen bien los acentos.
\usepackage{float} %Paquete para que funcione bien el comando \begin{figure}[H] -> ubicacion de figuras.
\usepackage{caratula}
\usepackage{amsmath} %Paquete para matematica - ie \begin{align*}
\usepackage{lastpage} %Permite hacer referencia a la cantidad de paginas.

% Normas tipográficas y opciones del español.
\usepackage[utf8]{inputenc} 

% Codificación de entrada (acentos).
\usepackage{graphicx}

% Variables de Titulo
\universidad{Instituto Tecnológico de Buenos Aires}
\facultad{Ingeniería Informática}
\departamento{Sistemas Operativos}
\materia{Trabajo Práctico 1}
\titulo{Filesystems, IPCs y Servidores Concurrentes}
\integrante{Castiglione, Gonzalo}{Legajo: 49138}
\integrante{Gomez, Horacio}{Legajo: 50825}
\integrante{Orsay, Juan Pablo}{Legajo: 49373}
\fecha{Segundo cuatrimestre de 2011}
\footspace{3cm}

% Otras Variables
\newtheorem{hip}{Hipótesis}

% Headers & Footers
\usepackage{fancyhdr}
\setlength{\headheight}{15pt} 
\pagestyle{fancy}

\fancyhead{} % Clear all header fields
\fancyhead[RE,LO]{Trabajo Práctico - 2do Cuatrimestre 2011}
\fancyhead[LE,RO]{\includegraphics[height=13pt]{logop.jpg}}
\fancyfoot{} % Clear all footer fields
\fancyfoot[LE,RO]{\thepage\ de \pageref{LastPage}}


% INICIO DEL DOCUMENTO
\begin{document}
	
\maketitle
\newpage

% Generar el indice
\setcounter{page}{1} %Resetea el contador para que no cuente la hoja del titulo.
\tableofcontents
\newpage

%\renewcommand{\abstractname}{Nuevo Titulo del Resumen}
\begin{abstract}

Para este trabajo se pedia la realizaciòn de una simulación de empresas, la cuales tienen aviones a su mando, encargados de distribuir sus medicamentos por una serie de ciudades con conecciones entre ellas limitadas.

\end{abstract}

\section{Modelo}

Para llegar a la estructura de la simulación actual, primero se estudio
la diferencia entre procesos y threads. Los threads presentaban la
ventaja de ser $light$ $weight$ y que no necesitan comunicación
mediante un $IPC$ para pasarse infromacion dado que corren en la
misma zona de codigo. Mientras que los procesos realizan una copia
de todas sus variables cuando son creados y corren indemendientes
de los cámbios que el proceso padre haga. Teniendo estos detalles
en cuenta, se decidió que cada $Compania$ debia ser un proceso, ya
que no tiene porque compartir su información con nadie mas. Y, dado
que los aviones comparten todo con su empresa, nos resulto conveniente
optar por $Threads$ para estos.

Al tomar este tipo de estructura, surgio el problema sobre típos de
mensajes entre estos procesos y cual era la forma óptima de comúnicarlos
dado que el pasaje de información es mucho mas complejo que la comunicación
entre threads, ya que requiere tanto sincronización como una zona
de memoria ya preparada para la lectura y escritura. \\


Tomada esta decision de diseño, se debia resolver además el problema
de:
\begin{enumerate}
\item Mostrar en pantalla los cámbios de cada $Compania$ luego de cada
turno.
\item Reflejar en las demás companias los cámbios producidos en el mapa
por la compaia $X$ antes que la compania $Y$ intente hacer un cámbio
sin haber recibido esta notificación.\\

\end{enumerate}
La primer solución propuesta fue la de crear un zona de memoria compartida
la que involucraria tener en cuenta los siguientes aspectos:
\begin{itemize}
\item Todo aquel proceso que quisiese modificar esta zona de memoria, tendría
que hacerlo siempre bloqueando previamente un mutex (o semaforo en
su defecto) y luego de realizados los cámbios liberar el mutex. Lo
cual obligaria a los demás procesos a {}``esperar'' en una cola
a acceder a esta memoria. El problema con esta implementación es que
si se hubiese implementado, no se se hubiese respetando la consigna
de usar los diferentes típos de ipcs para la comunicación entre procesos.
Sin embargo es una solución agradable ya que evitaba el tener que
andar notificando a otras companias mediante paquetes especiales sobre
cámbios realizados.
\end{itemize}
La segunda solución propuesta involucaraba un proceso $servidor$
que se encargaría de la administración de turnos y recursos para cada
compania.
\begin{itemize}
\item Este presentaba la ventaja de tener una implementación muy sencilla
ya que solamente se ocuparía de levantar semaforos y bajarlos para
que las companias toquen el mapa en forma sincronizad y además asegurar
que ningúna valla a jugar dos veces seguidas (luego veremos que, como
en toda buena idea, trajo sus complicaciones).

\begin{itemize}
\item A su vez se podia saber cuando todas las companias habían terminado
un turno; por lo que actualizar la UI por turno resultaria muy sencillo.
\item La única contra que encontramos es que la comunicación con las companias
se volvia un tanto compleja.
\end{itemize}
\end{itemize}
\begin{figure}[H]
\hspace{1cm}\includegraphics[scale=0.5]{imagenes/design.png}

\caption{Estructura de procesos y threads de la simulación}


Al iniciar el programa, lo primero que se realiza es el parseo de
la información (de companias, mapas), inicializacion del entorno (IPCs,
semaforos, handlers de las señales, inicializacion de los nuevos procesos...)
y luego de terminado todo esto, se procede a inicializar el proceso
de UI/Servidor.
\end{figure}


Uno de los primeros (y mas grandes) problemas que se presentaron al
aplicar este diseño era de como reflejar los cámbios hechos por una
compania en todas las demás. Inicialmente, se decidió que el servidor
tendría una (y la única) instancia del mapa y que se pasaría al principio
del turno a la compania, esta lo modificaria y luego se lo pasaría
nuevamente al servidor con los cámbios. Y asi para cada compania. 

Pero esta solución era muy costosa, ya que además de tener que mandar
el mapa, se debia mostrar por pantalla (en forma ordenada) toda la
información de la empresa; por lo que además de tener que pasar el
mapa dos veces, se tendría que sumar toda la compania, lo cual implicaria
muchísimo procesamiento y uso de memoria que podria ser ahorrado!.

Lo que nos llevo a proponer una segunda solución alternativa, la cual
implicaba que tanto el servidor como las companias tendrían una instancia
del mapa(inicial) y este se iría actualizando mediante paquetes $updates$
que se enviarian desde el servidor luego de cada turno. 

Esto presenta la ventaja que la comunicación entre los procesos se
reduciría a únicamente sus cámbios! pero la contra esta en que requeria
de una clase $serializer$ muchísimo mas completa que la que se tenia
en mente. Sin embargo, luego de discutirlo se llego a que se esta
era la mejor implementación %
\footnote{No se presentan calculos de cuanto mas se mejora dado que es muy fácil
notar la cantidad de infromacion que se ahorra por cada envio de información%
}\\


Teniendo en cuenta todas estas consideraciones, la lógica del $servidor$
seguiría el siguiente comportamiento:
\begin{verbatim}
Por cada compania:

Se le da un turno.

Se espera a que finalize su turno.

Se hace un broadcasting de los updates leidos a 

   todas las demás companias.

Actualizar UI. \\

\end{verbatim}
De esta manera, resulta fácil imaginar la lógica de una $compania$:
\begin{verbatim}
Despierto a todos mis aviones.

Cuando todos movieron, actualizo a un nuevo target a todos

    aquellos que llegaron a destino.

Escribo los paquetes con los cámbios realizados por cada uno al servidor.

Me escribo a mi misma.
\end{verbatim}
A continuación de presenta un esquema de como se penso a una compania:

\begin{figure}[H]
\hspace{1cm}\includegraphics[scale=0.5]{imagenes/airlineModl.png}

\caption{Esquma de una compania}
(figura tomada del sitio: http://www.chuidiang.com)
\end{figure}


Tal como se explico anteriormente, cada $Thread$ representa a un
avion activo, en donde este tiene su memoria propia (items, posicion,
target, ...), una memoria compartida (el mapa) y un comportamiento.
Cuando un avion no tiene movimientos posibles, se mata al proceso,
y este ciclo sigue hasta que ningun avion tenga movimientos disponibles.
En cuyo caso se envia un paquete de tipo $company$ $status$ $update$
al servidor y se encargara de hacer lo que sea necesario.

\pagebreak{}


\section{IPC}

Al principio se comenzó experimentando comunicación entre procesos
con pipes. Luego de varias complicaciones de llego a una primera version
de metodos de IPC formada por diversos metodos. Esta nos sirvio como
base para ir experimentando como funcionaban los semaforos y mutex
en linux. Luego, a medida que el programa iba tomando forma y sentido,
se comenzó a mejorarla y se implementó $Fifos$, $MsgQueues$, $Sockets$
y $Shared$ $Memory$.

Uno de los mayores inconvenientes al momento de implementación de
los IPCS era la sincronización, ya que, por mas pruebas que se les
halla a cada antes de ser montados al codigo de la simuacion, se tenia
el problema que cuando no se leia en forma ordenada, este decia que
no habia mensajes cuando se espereaba que los hubiera, lo que llevaba
a mucha perdida de tiempo en debugeo de codigo.

Una de las complicaciones que tuvimos con los IPCs era el tamaño de
los datos a escribir, dado que para algunos IPCs era mas fácil trabajar
con tamaños fijos y con otros daba igual, se opto por paquetes de
tamaño fijo. Esto fácilito mucho la programacion de la interfaz de
comunicación, pero tiene el problema importante que si se quiere serializar
una gran compania con muchos aviones e items, puede resultar que este
string serializado resultante sea mas largo que el tamaño default.
Lo que provocaria que el paquete no se mande y quede inconsistente
el resultado de la simulación. 

De todas maneras, si esto llegase a suceder, un mensade de error se
mostrara en el $log.txt$. En cuyo caso lo único que se tiene que
arreglar es el $DATA$\_$SIZE$ definido en $communicator.h$.

Otro inconventiente que trae esta decicion que tamaño fijo para mensajes
es el desperdicio de memoria innecesario por ejemplo para mandar un
pequeño update. Sin embargo, no es un gran problema, si el dia de
mañana se necesitase que esto no sea asi, bastara con simplemente
cambiar la implementación del IPC correspondiente (no afectaría de
ningúna manera a la simulación).\\



\subsection{Shared Memory}

La implementación que actualmente se tiene para shared memory, es
aprovechar el hecho que los tamaños de los mensajes son fijos y reservar
en memoria bloques con un tamaño de $M$ mensajes por cada proceso
que quiera escribir. Es decir, dividimos un pedazo fijo de memoria
en $NxN$ lugares, en donde $N$ esta dado por la cantidad de luagres
especificados en $ipc$\_$init$. Esta implementación ofrece la ventaja
de ser muy eficiente en cuando acceso a un mensaje, ya que tiene una
complejiad $o(1)$, pero la gran contra es que es mucha la memoria
que necesita tener reservada.

Dentro de cada lugar de esta matriz, se tiene un array con los $M$
luagres ya mencionados y adelante de todo un $char$ indicando cuantos
se tienen en la lista.


\subsection{Típos de paquetes}

Actualmente la comunicación entre $servidor\Longleftrightarrow compania$
se realiza mediante tres típos de paquetes:\\

\begin{itemize}
\item City update: Este paquete contiene información de que item fue modificado
en que ciudad y en que cantidad. Si muchas ciudades fueron modificadas
en un turno, uno por cada cámbio va a ser enviado al servidor.
\item City status update: Este paquete se envia cunado una compania detecto
que sus aviones no pueden abastecer a mas ciudades y quiere {}``darse
de baja'' en el servidor. Es importante esto ya que, una vez que
una compania se desactiva en el servdor, automaticamante se deja de
encolarle actualizaciones. Lo que resulta es un ahorro importante
en llamadas al serializer y de uso de recursos del IPC para mantener
mensajes que nunca serial leidos.
\item Company update: este es el paquete mas grande que se tiene, aqui es
donde se guarda la infromacion de todos los aviones que tiene una
compania. El problema es que crece mucho a medida que se agregan mas
items y/o aviones la misma. Por lo que quedaria en el $"wish$ $list"$,
partir este paquete en otros mas pequeños y en lo posible con tamaño
constante, para de esta manera poder reducir el DATA\_SIZE que se
necesita tener reservado para usar el IPC. Que en estos momentos seria
lo que mas recursos consume de la simulación. Se tenia pensando para
estos, realizar lo mismo que para las ciudades, solamente mandar pequelos
paquetes updates en vez de repetir una y otra vez la misma infromacion
que no sufrio camibios.
\end{itemize}
\pagebreak{}


\section{Otras consideraciones}

Uno de los problemas mas dificiles que constantemente se presentaba
durante el desarrollo del trabajo era sobre la dura eleccion entre
$tiempo$ $vs$ $memoria$. Si se debia hacer cierto calculo al momento
de neceitarlo o bien guardarlo previemente y simplemente actualizarlo
cuando sea necsario.

Al principio, cada vez que un avion llegaba a una ciudad, se realizaba
un DFS para encontrar el camino a las ciudad mas corta que sea abastecible.
El problema es que este mismo DFS se estaba realizando por cada avion,
por cada compania en cada turno. Lo que nos parecia que se realizaba
el mismo calculo innecesariemente muchisimas veces. Por lo que se
propuso una matriz en donde, por cada compania se colocaria un indice
diciendo a donde hay que ir para acceder a la ciudad $X$ (para cualquier
ciudad) y a que distancia se encuentra. Esto reduciría la llamda al
$DFS$ a desrefenciar una matriz ($o(1)$), esta fue una de las luchas
mencionadas anteriormente, sobre la ultiliacion de memoria espacial
o de tiempo%
\footnote{Esta razon de necesitar implementar un DFS surgio en casos cuando
solo se podia abastecer a uan ciudad que no era adyacente a la posicion
actual del avion.%
}.

Otro cámbio de la misma indole que se propuso durante la realizcion
del trabajo es que cada ID de un elemento representaba a la pocision
en el array que ocuparía. Es decir, el elemento con Id $5$, se lo
encontraria en el array de elementos en la quinta posicion. Esto trae
consecuencias como tener un array de $20$ elementos solo para tener
un elemento con ID $20$. Sin embargo, para la busqueda de items (la
cual se reliza constantemente) reduce un algortimo de $o(n)$ a uno
de $o(1)$ con el precia de agregar un extra en la cantidad de memoria
necesaria para su funcionamiento.

Se tuvo un problema de ultimo momento con dos de los $IPCs$ (Sockets
y Fifos) que dejaron de andar y se trataron de arreglar pero no se
pudo testear si verdaderamente tienen una funcionalidad $100\%$ correcta. 

El problema con sockets es que era necesario mantener una referencia al file 
descriptor del socket de lectura y eso no lo descubrimos hasta ultimo momento.
Nos hubiese dado mucho gusto encontrar la solución  de esto pero nos quedamos 
sin tiempo para la entrega. Por lo que se tuvo que 
implementar un array de tamaño fijo para almacenar estos files descriptos.
Queda en la lista arreglar eso y hacerlo independiente de la implementacion.


\subsection{Otras problemas}

Un problema al que todavía no pudimos encontrar explicacion es porque
los semaforos de $System$ $V$ a veces no se inicializaban para ciertos
valores de claves. Se intento eliminarlos, cambiar los flags de creacion,
reiniciar la comutadora pero todo sin suerte! Mas raro resultaba aun
que al sumarle un magic number al key, estos se creban y sin problema.
Por lo que nuestra única alternativa fue la de pasar a semaforos de
Posix. Los cuales resultaron increiblemente mas simples. Lo mismo
sucede a veces con la creacion de los $IPCs$, que a veces no logran
cerrarse correctamente al finalizar el programa o bien el recuso del
SO se encuentra ocupado y se rehusa a crear los IPS terminando en
resultados sin sentido, por lo que se recomienda que si esto sucede
se utilize el comando $ipcs$ para verificar que no se tenga una excesiva
cantidad de estos andando (se solucióna tambien reiniciando ya que
para todo tipo de almacenamiento temproal, se utiliza la carpeta $/tmp$
del SO.

Se queria aclarar que si se testea la simulación de archivos que tienen
formatos invalidos, el programa resultara en $segmentation$ $fault$.
Esto no fue arreglado ya que supusimos que lo importante de este trabajo
practico no era ver quien hacia el parser mas robusto sino el de demostrar
el poder del uso de IPCs. Por lo que se recomienda verificar bien
el formato de los archivos antes de iniciar el programa.

\pagebreak{}


\section{Concusiones}

Luego de realizado este trabajo practico, se aprendio sobre la potencia
a la que se puede llevar un programa al hacerlos que sean capaces
de dividir sus tareas en proecsos concretos y por sobre todo, al mantenerlos
sincronizados. 
Existen diferentes formas para comunicar procesos:
\begin{itemize}
 \item socket: es el único que permite sincronización entre diferentes máquinas. También
utilizan archivos especiales del tipo socket para trabajar localmente.
 \item shared memory: es veloz y mucho más dificil de implementar, hay que manejar la
memoria a mano y a menos que se utilzen librerias como $OSSP mm$ (Shared Memory Allocation)
no se puede alocar memoria (malloc) dentro del segmento compartido.
 \item fifo: crea tipos especiales de archivo en el filesystem también conocidos como
''named pipes'', de facil utilización.
 \item msg queue: similar a fifo, utiliza colas para almacenar mensajes con keys para
acceder a ellas.
\end{itemize}
Es necesario sin embargo manejar la sincronización de los mencionados ipc para que los
diferentes procesos que acceden a ellos no dejen inconsistencias en la comunicación.

\end{document} 
